\chapter{Einleitung}
\label{chap:einleitung}
In diesem Kapitel werden zunächst die Grundlagen erläutert, welche für das Verständnis dieser Arbeit notwendig sind. Außerdem werden in den Grundlagen alle wichtigen Begriffe erklärt, die zum Verständnis des Themas beitragen und notwendig sind. Anschließend wird auf die zugrundeliegende Problemstellung eingegangen und darauf aufbauend auf das Ziel der Arbeit.

\section{Grundlagen}
\label{sec:grundlagen}
Grundsätzlich ist das in dieser Arbeit behandelnde Thema für jede Person mit einer allgemeinen Informatikausbildung ohne weiteres zu verstehen. Es kann bei dieser Personengruppe, die Kenntnisse über grundsätzlichen Prozess einer Softwareentwicklung vorausgesetzt werden. Trotzdem soll im weiteren Verlauf einige Begriffe genauer erklärt werden.
\\\\
\textbf{Delivery}\\
Delivery (zu deutsch Ausliefern) beschreibt das Ausliefern (das Verteilen) von Artefakten. Dabei kann das Artefakt eine ganze Applikation oder nur ein Service in einer Service Orientierten Architektur sein.
\\\\
\textbf{Deployment}\\
Unter Deployment (zu deutsch Softwareverteilung) versteht man das installieren, eines Artefaktes. Auch hier kann ein Artefakt eine ganze Applikation oder nur ein Service sein.

\section{Problemstellung}
\label{sec:problemstellung}
Durch fortschreitende Technologien, werden Arbeitsabläufe immer automatisierter. Dadurch wird die Zeitspanne für \gls{glos:TTM} immer relevanter und kürzer, wodurch der Wettbewerbsdruck wächst. Daher ist es wichtig eine kurze TTM zu haben.

Eberhard Wolff beschreibt in \cite[S. 2 ff.]{EWolff:CD} einen Fall eines fiktiven E-Commerce Unternehmens.
Das Unternehmen hatte nur eine große Software, den E-Commerce Shop. Durch neue Angebote und das dauerhaft ändernde Interesse der Kunden mussten neue Funktionen regelmäßig und in möglichst kurzen abständen dem Kunden zugänglich gemacht werden. Dies wurde jedoch durch die Tatsache behindert, dass die Software über die Jahre gewachsen ist und das erneute Ausliefern der Software für eine Funktion sich nicht lohnte. Daher wurde nur einmal im Monat neu Deployed. Der Prozess wurde außerdem dadurch behindert, dass die Qualitätssicherung zwar ein Teil der Softwareentwicklung war, jedoch Tests nur manuell ausgeführt worden sind, wodurch regelmäßig Fehler übersehen wurden.

die Software wurde schließlich mit Fehlern ausgeliefert und es stellte sich erst am nächsten Tag, oder schlimmer nach einer Woche, heraus, dass sie nicht einwandfrei funktionierte. Entwickler mussten also ihre Arbeit unterbrechen und den Fehler finden und beheben. Da jedoch ein wenig Zeit vergangen ist, seit dem die Entwickler an diesem Teil des Codes gearbeitet haben, müssen sie sich erst wieder einarbeitet, bis sie den Fehler finden und beheben können.

Das Unternehmen hatte eine große TTM-Zeit und dadurch hohe kosten. Zusätzlich entstehen fehlerhafte Releases wodurch zusätzliche Kosten bzw. Einbußen entstehen.

\section{Ziel der Arbeit}
\label{sec:zielDerArbeit}
In dieser Arbeit soll Continuous Delivery genauer erläutert und dabei folgende Leitfragen beantwortet werden:

\begin{enumerate}
	\item Was ist Continuous Delivery?
    \item Was ist eine Continuous Delivery Pipeline?
	\item Wie kann man Continuous Delivery in ein bestehenden Entwicklungsprozess einbinden?
\end{enumerate}

Die erste Leitfrage soll den Begriff Continuous Delivery und seine Herkunft erläutert. Dabei wird kurz auf die Geschichte der Softwareentwicklungsprozesse eingegangen und erläutert wie sich der Prozess zum heutigen unterscheidet. Außerdem wird in diesem Zusammenhang noch einmal erläutert, warum sich die Prozesse verändert haben bzw. verändert werden mussten.

Darauf aufbauend, soll eine Continuous Delivery Pipeline erläutert und aufgebaut werden, anhand dieser wird mit der folgenden und abschließenden Leitfrage, ein bestehender Entwicklungsprozess in Continuous Delivery eingebunden.