\chapter{Einleitung}
\label{chap:einleitung}
In diesem Kapitel werden zunächst die Grundlagen erläutert, welche für das Verständnis dieser Arbeit notwendig sind. Außerdem werden in den Grundlagen alle wichtigen Begriffe erklärt, die zum Verständnis des Themas beitragen und notwendig sind. Anschließend wird auf die zugrundeliegende Problemstellung eingegangen und darauf aufbauend auf das Ziel der Arbeit.

\section{Grundlagen}
\label{sec:grundlagen}
Grundsätzlich ist das in dieser Arbeit behandelnde Thema für jede Person mit einer allgemeinen Informatikausbildung ohne weiteres zu verstehen. Es kann bei dieser Personengruppe, die Kenntnisse über grundsätzlichen Prozess einer Softwareentwicklung vorausgesetzt werden. Trotzdem soll im weiteren Verlauf einige Begriffe erläutert und den Prozess der Softwareentwicklung genauer erklärt werden.
\\\\
\textbf{Softwareentwicklung}\\
% ZITIEREN
Im allgemeinen wird Softwareentwicklung als ein Prozess zur Erstellung von Software verstanden, welche folgende Phase beinhaltet:
\begin{enumerate}
	\item Planung
	\item Analyse
	\item Entwurf
	\item Implementierung
	\item Validierung und Verifikation
	\item Abnahme
	\item Release
\end{enumerate}
Im Rahmen dieser Arbeit wird der Begriff Softwareentwicklung jedoch als Synonym für die Phasen vier bis sieben genommen. Kapitel \ref{sec:problemstellung} \nameref{sec:problemstellung} verdeutlicht noch einmal genauer, die Relevanz dieser Phasen.
\\\\
\textbf{Qualität}\\
Qualität ist nach der DIN 55350 wie folgt definiert: "Qualität ist die Beschaffenheit einer Einheit bezüglich ihrer Eignung, festgelegte und vorausgesetzte Erfordernisse zu erfüllen. Dabei wird ergänzend die Einheit als materieller oder immaterieller Gegenstand der Betrachtung und die Beschaffenheit als Gesamtheit der Merkmale und Merkmalswerte definiert."
\\\\
\textbf{Qualitätssicherung}\\
Die Qualitätssicherung sollte ein in die Softwareentwicklung integrierter Prozess sein, um die Einhaltung der für das Projekt festgelegten Qualitätsmerkmale zu überprüfen und so die Qualität des entstehenden Produktes zu gewährleisten. Zur Überprüfung dieser Merkmale können verschiedene Werkzeuge eingesetzt werden, die in Kapitel \ref{chap:continuousDelivery} \nameref{chap:continuousDeliver} genauer erläutert werden.

\section{Problemstellung}
\label{sec:problemstellung}


\section{Ziel der Arbeit}
\label{sec:zielDerArbeit}