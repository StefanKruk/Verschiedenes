\chapter{Systematische Literaturrecherche}
\label{chap:sysRecherche}
Die Systematische Literaturrecherche stellt die Basis der Quellen und Informationen, des in Kapitel \ref{chap:continuousDelivery}\ \nameref{chap:continuousDelivery}\ vorgestellten Inhalts dar.

In diesem Kapitel wird daher aufgezeigt, wie die herangezogenen Quellen und Informationen ermittelt, welche Auswahl- und Ausschlusskriterien festgelegt wurden und was für Ergebnisse die entsprechenden Suchanfragen gebracht haben.

\section{Auswahlkriterien und Suchbegriffe}
\label{sec:auswahlkriterienUndSuchbegriffe}
Um die Auswahl der Literaturen zu Filtern, wird zunächst allgemeine Auswahl- und Ausschlusskriterien definiert, mit denen die im \nameref{sec:rechercheprotokoll} angegebenen Suchergebnisse begründet werden.
Anschließend wird der Zugrundlegende Ansatz (P.I.C.O.C.) genauer erläutert und darauf aufbauend Suchbegriffe in Deutsch und Englisch definiert.

\subsection{Inhaltliche Auswahlkriterien}
\label{subsec:inhaltlicheAuswahlkriterien}
Folgende Inhaltliche Auswahlkriterien wurden für die Recherche festgelegt:
\begin{list}{label}{spacing}
	\item[\textbf{P1}] Dokument ist über oder hat direkten Bezug zu Continuous Delivery
	\item[\textbf{P2}] Dokument beschreibt die Einsatzmöglichkeiten von Continuous Delivery
	\item[\textbf{P3}] Dokument beschreibt wichtige Technologien für den Einsatz von Continuous Delivery
	\item[\textbf{P4}] Dokument kann zum beantworten der Leitfragen genutzt werden.
\end{list}
Mit Hilfe der Auswahlkriterien wird im \nameref{sec:rechercheprotokoll} die Relevanz der gefundenen Materialien begründet. Sie werden über die Buchstaben a) bis c) referenziert.
\subsection{Inhaltliche Ausschlusskriterien}
\label{subsec:inhaltlicheAusschlusskriterien}
Folgende Inhaltliche Ausschlusskriterien wurden für die Recherche festgelegt.
\begin{list}{label}{spacing}
	\item[\textbf{N1}] Dokument ist zu allgemein und/oder hat nur am Rande etwas mit dem Thema zu tun (Bsp.: Enthält den Begriff nur in Referenzen)
	\item[\textbf{N2}] Dokument ist unvollständig.
	\item[\textbf{N3}] Inhalt des Dokumentes muss kostenpflichtig erworben werden oder ist aus anderen Gründen nicht einsehbar
	\item[\textbf{N4}] Inhaltsangabe, Einleitung, Fazit oder Abstract sind nicht Aussagekräftig bzw. lassen keine Hinweise auf den Einsatz oder der Beschreibung von Continuous Delivery zu
	\item[\textbf{N5}] Inhalt trägt nicht zur Beantwortung der Leitfragen bei. (Bsp.: Das Dokument ist ein Erfahrungsbericht, enthält jedoch keine konkreten Erläuterungen oder Verweise auf Dokumente, die den Prozess genauer beschreiben.)
	\item[\textbf{N6}] Dokument ist nicht in Deutsch oder Englisch (Material kann aufgrund der Sprachbarriere nicht verwendet werden)
	\item[\textbf{N7}] Inhalt des Dokumentes muss kostenpflichtig erworben werden oder ist aus anderen Gründen nicht einsehbar.
	\item[\textbf{N8}] Dokument beschreibt wie ein Werkzeug und/oder Framework aufgebaut ist und funktioniert.
\end{list}
Mit Hilfe der Ausschlusskriterien wird im \nameref{sec:rechercheprotokoll} die Irrelevanz der gefundenen Materialien begründet. Sie werden über die Buchstaben d) bis j) referenziert.

\subsection{P.I.C.O.C.}
\label{subsec:picoc}
Die Suchbegriffe werden mit Hilfe des PICOC-Ansatzes (siehe \cite{Kitchenham2007}) ermittelt.

\subsubsection*{Population}
\label{picoc:population}
Die Population, zu Deutsch etwa "Bevölkerung", beschreibt eine Teilmenge von relevanten Personen, wie Tester, Manager, Novizen oder Experten. Aber auch Applikationsfelder wie IT-Systeme, Command und Control Systeme oder Industrielle Gruppen wie Telekommunikations- oder kleine IT-Unternehmen.
\\\\
Bezüglich der Population werden hier die sogenannten "Stackeholder" weggelassen. Da das Thema jedoch stark mit dem Thema DevOp-Teams tangiert, wird hier zusätzlich zu diesem Thema Suchanfragen gestellt. Eine Einschränkung des Thema auf DevOps besteht jedoch nicht.

\subsubsection*{Intervention}
\label{picoc:Interventation}
Bei der Intervention handelt es sich um die eingesetzten Methoden, Werkzeugen, Technologien oder Prozeduren für ein bestimmtes Problem. Im Rahmen von Continuous Delivery bedeutet dies, die Werkzeuge die nötig sind um eine Continuous Delivery Pipeline aufbauen und durchführen zu können. Grundlegend sollen Werkzeuge erläutert werden, welche zur Verwaltung, Bauen, Testen und Ausliefern dienen.

\subsubsection*{Comparison / Vergleich}
\label{picoc:Comparison}
Der Vergleich (Comparison) beschreibt die Werkzeuge einer Kontrollgruppe, welche mit denen aus der Intervention verglichen werden. Konkret werden die Gruppen aufgrund der TTM verglichen. Anzumerken sei, dass die Vergleichsgruppe ohne einen automatisierten Prozess Softwareprojekte durchführt.

\subsubsection*{Outcomes / Auswirkung}
\label{picoc:Outcomes}
Bei der Auswirkung soll mit Hilfe von Zahlen und Faktoren der Vergleich erläutert werden zwischen der Intervention und der Kontrollgruppe. Hier kann zum Beispiel die Zeitspanne des Time-to-Market verglichen werden, was wiederum einen Einblick in die Kosten für die Implementierung/Auslieferung eines Produktes/einer Funktion gibt.

\subsubsection*{Context / Kontext}
\label{picoc:Context}
Der Kontext ist hier die Software-Entwicklung in Bezug auf Aufwand und Kosten der Produktion. Bei Continuous Delivery spielt ebenfalls der Aufbau des Teams eine Rolle. Wie bei der \nameref{picoc:population} spielen hier DevOp-Teams eine zentrale Rolle. Zusätzlich sei hier auf \nameref{picoc:Interventation} und die dortigen Werkzeuge verwiesen.

\subsection{Suchbegriffe}
\label{subsec:suchbegriffe}
Im Nachfolgenden sind für jeden, der in Abschnitt \ref{subsec:picoc} \nameref{subsec:picoc} genannten Aspekte Synonyme in Deutsch und Englisch festgehalten, welche die Basis für die verwendeten Suchanfragen bilden.
\\\\
\begin{tabular}{|l|l|l|}
	\hline 
	\rowcolor{listinggray} \textbf{Kategorie} & \textbf{Deutsche Begriffe} & \textbf{Englische Begriffe} \\ 
	\hline
	Population & \multicolumn{2}{c|}{DevOps} \\ 
	\hline
	Intervention & \multicolumn{2}{p{7cm}|}{
				\begin{itemize}
					\itemsep-15pt
					\item Continuous Delivery
					\item Docker
					\item Jenkins
					\item Deployment
                    \item pipeline
                    \item DevOPs
				\end{itemize}
			} \\ 
	\hline
	Comparison &  \multicolumn{2}{p{10cm}|}{
		\begin{itemize}
			\itemsep-15pt
			\item Manuel (Deployment)
		\end{itemize}
	} \\  
	\hline
	Outcomes & \multicolumn{1}{p{5cm}|}{
		\begin{itemize}
			\itemsep-15pt
			\item Kosten
			\item Tests
			\item Zeit
		\end{itemize}}  &  \multicolumn{1}{p{5cm}|}{
		\begin{itemize}
			\itemsep-15pt
			\item costs
			\item tests
			\item time / duration
		\end{itemize}} \\
	\hline
	Context & \multicolumn{1}{p{5cm}|}{
		\begin{itemize}
			\itemsep-15pt
			\item Technik
			\item Prinzipien
			\item Praxis
			\item Anwendung
		\end{itemize}}  &  \multicolumn{1}{p{5cm}|}{
		\begin{itemize}
			\itemsep-15pt
			\item techniques
			\item principles
			\item practice
			\item usage
		\end{itemize}} \\
	\hline
\end{tabular} 
\noindent
Der Begriff Deployment is sowohl in Intervention und Comparison, da dieser ein zentraler Begriff in beiden Mengen ist und unterschiedlich verwendet werden kann.

\subsection{Zusätzliche Anmerkungen zur Recherche}
\label{subsec:zusätzlicheAnmerkungenZurRecherce}
Für diese Arbeit wurden zusätzlich folgenden Einschränkungen, für die Durchführung der systematischen Suche festgelegt:

\begin{description}
	\item[Zugänglichkeit] Materialien müssen entweder öffentlich oder für den Personenkreis, für die diese Arbeit angefertigt wird, ohne weitere Einschränkungen, wie ein notwendiges Login, zugänglich sein. da ansonsten der Beschaffungsaufwand zu hoch ist und die Materialien nicht für andere Studenten bzw. den Dozenten zugänglich wären.
	\item[Auswahl der Materialien] Materialien werden anhand der Inhaltsübersicht, Einleitung, Fazit oder eines Abstracts ausgewählt, da ein einlesen in einzelne Kapitel zu viel Zeit beanspruchen würde.
	\item[Suchergebnisse] Bei auffinden von Großen Mengen bei der Suche, wird zunächst versucht durch eventuelle Filtermöglichkeiten, die Relevanz der Materialien zu ordnen und die ersten 20 Resultate begutachtet. Dabei wird die Qualität der Ordnung und die Effizienz des zugrunde liegenden Algorithmus der Suchmaschine überlassen. Sollten keine Filtermöglichkeiten vorhanden sein, wird anhand der Kurzbeschreibungen und der Titel die ersten 20 besten Treffer ausgewählt.
\end{description}

\section{Quellen und Suchanfragen}
\label{sec:QuellenUndSuchanfragen}

\subsection{Suchstring}
\label{subsec:suchstring}
In Abschnitt \ref{subsec:suchbegriffe} \nameref{subsec:suchbegriffe} wurden Suchbegriffe festgelegt, auf denen die Literaturrecherche basiert. Diese werden zu nächst mit einem "'ODER" bzw. "'OR"` verknüpft. Im Zweiten Schritt werden die Suchbegriffe mit einem "'UND`" bzw. "AND"` verknüpft.
\\\\
Da bestimmte Begriffe mit verschiedenen Kontexte in Verbindung gebracht werden können. Wird zunächst ein Suchstring aufgebaut, der als Erstes Element ("'Continuous Delivery"` OR "'Deployment"`) besitzt. Nachfolgend werden nun die einzelnen Suchstrings aufgelistet, die verwendet wurden, um die systematische Literaturrecherche durchzuführen. Da Suchmaschinen einen komplexen Algorithmus aufweisen, wird ebenfalls davon ausgegangen, dass auch eine aneinander reihen von den in \ref{subsec:suchstring}\ \nameref{subsec:suchstring}\ Ausdrücken (immer mit dem Führenden "'Continuous Delivery"` Oder "'Deployment`") zum gewünschten Ergebnis führt. Dies beruht auf den Eigenschaften moderner Suchalgorithmen.
Nach einer ersten suche, hat sich ergeben, dass einige Begriffe irreführend für die Suchmaschinen sind, wodurch Materialien gefunden wurde, welche absolut nichts mit dem Thema zu tun haben. Daher sind nur die folgenden Suchanfragen von Bedeutung.

\begin{description}
	\item[\suchstring{Komplex} =] \suchstringPrefix \\AND\\ ("'SSH"` OR "'FTP"` OR "'Deployment"` OR"'Kosten"` OR "'costs"` OR "'Tests"` OR "'Zeit"` OR "'time"` OR "'duration"` OR "'Fehler"` OR "'error"` OR "'Qualitätssicherung"` OR "'quality assurance"` OR "'Technik"` OR "'techniques"` OR "'Prinzipien"` OR "'principles"` OR "'Praxis"` OR "'practice"` OR "'Anwendung"` OR "'usage")
    
    \item[\suchstring{deployment} =] \suchstringPrefix \\AND\\ ("'Deployment"`)
    
    \item[\suchstring{pipeline} =] \suchstringPrefix \\AND\\ ("'Pipeline"`)
    
    \item[\suchstring{devops}] \suchstringPrefix \\AND\\ ("'DevOps"`)
\end{description}\noindent

Da nach \cite[vgl. S. 26]{Kitchenham2007} auch ein einfacher Suchstring effektiv sein kann, wurde neben den bereits genannten Suchstrings noch ein weiterer abgeleitet, welcher lediglich aus dem Thema dieser Arbeit besteht.

\begin{description}
	\item[\suchstring{einfach}] "Continuous Delivery"
\end{description}\noindent
Dieser Suchstring wird ausschließlich dazu verwendet, bei einer Suche, den Dokumenten bzw. den Publikations Titel, nach vorkommen dieser Begriffe zu durchsuchen.


\subsection{Quellen}
\label{subsec:quellen}
In \cite{Kitchenham2007} wurden einige elektronische Standartquellen der Informatik genannt. Nachfolgend, werden diese noch einmal aufgelistet und kurz begründet, warum jeweilige Quellen gewählt bzw. nicht gewählt wurde.

\subsubsection{ACM Digital Library}
Dokumente aus der \textit{ACM Digital Library} sind nicht frei zugänglich und müssen zunächst erworben werden, bevor sie gelesen werden können.

\subsubsection{Citeseer Library}
Eine erste Literaturrecherche in der \textit{Citeseer Library} ergab keine nennenswerten Einträge. Es wurden Dokumente zu Themen zurückgeliefert, welche nur ansatzweise etwas mit dem Thema zu tun haben und nach \textbf{N5} ausscheiden.

\subsubsection{EI Compendix}
Bei \textit{EI Compendix} ist zunächst eine Registrierung erforderlich, bevor Dokumente angesehen werden können.

\subsubsection{IEEExplore}
Bei IEEExplore handelt es sich um eine Digitale Datenbank für wissenschaftliche Literaturen. Sie ist für jeden Informatiker, bzw. für jedes Informatik-Thema eine wichtige Ressource für die Literaturrecherche. Sie ist einer der wichtigsten Datenbanken für wissenschaftliche Arbeiten im Bereich Informatik, Elektrotechnik und Elektronik.
\\\\
Informationen bezogen von: http://ieeexplore.ieee.org/xpl/aboutUs.jsp

\subsubsection{Inspect}
Bei \textit{Inspect} ist zunächst eine Registrierung erforderlich, bevor Dokumente angesehen werden können.

\subsubsection{Google Scholar}
\textit{Google Scholar} hat ein großes Angebot und die \underline{meisten} PDFs sind frei zugänglich. Jedoch ist hier zu beachten, dass \textit{Google Scholar} auch auf Dokumente verweist, die zum Beispiel im \textit{IEEEXplore} liegen.

\subsubsection{ScienceDirect}
Die Dokumente in \textit{ScienceDirect} sind nur teilweise frei zugänglich.

\subsection{Gewählte Quellen}
Nachstehende Tabelle zeigt noch einmal die gewählten und nicht gewählten Quellen:
\\\\
\begin{tabular}[]{|L{10cm}|l|}
	\hline
	\rowcolor{listinggray} \textbf{Quelle} & \textbf{gewählt}  \\ 
	\hline
	ACM Digital Library & \redX \\ 
	\hline
	Citeseer Library (citiseer.ist.psu.edu) & \redX  \\ 
	\hline
	EI Compendex (www.engineeringvillage2.org) & \redX  \\ 
	\hline
	IEEExplore & \greenchecked  \\ 
	\hline
	Inspec (www.iee.org/Publish/INSPEC/) & \redX \\ 
	\hline
	Google scholar (scholar.google.com) & \greenchecked \\ 
	\hline
	ScienceDirect (www.sciencedirect.com) & \redX  \\
	\hline
\end{tabular} 

\section{Ergebnisse der Recherche}
\label{sec:ergebnisseDerRecherche}
In der Nachfolgenden Tabelle sind, nach Suchmaschine geordnet, die Ergebnisse der systematischen Literaturrecherche. Es wird der Suchstring, die Anzahl der gefundenen Ergebnisse, die Anzahl er betrachteten Ergebnisse, die Anzahl der relevanten Ergebnisse, die Anzahl der gewählten Ergebnisse und das Datum an dem die Suche durchgeführt worden ist.
\\
\begin{tabular}{|l|l|l|l|l|l|}
\hline
     \rowcolor{listinggray}\multicolumn{6}{|l|}{\textbf{IEEExplore}}  \\ 
\hline
    \rowcolor{listinggray}Suchstring & gesamt & betrachtet & relevant & gewählt & Datum \\
    \hline 
    \suchstring{komplex} & 201.863 & 200 & 0 & 0 & 23.06.2016 \\
    \hline
    \suchstring{deployment} & 12.092 & 200 & 1 & 1 & 23.06.2016 \\
    \hline
    \suchstring{pipeline} & 10 & 10 & 6 & 6 & 23.06.2016 \\
    \hline
    \suchstring{devops} & 11 & 11 & 4 & 1 & 23.06.2016 \\
    \hline
    \suchstring{einfach} & 40 & 40 & 10 & 6 & 23.06.2016 \\
    \hline
\end{tabular}
\\[20px] \noindent     
\begin{tabular}{|l|l|l|l|l|l|}
	\hline
    \rowcolor{listinggray}\multicolumn{6}{|l|}{\textbf{Google scholar}}  \\
	\hline
    \rowcolor{listinggray}Suchstring & gesamt & betrachtet & relevant & gewählt & Datum \\
    \hline 
    \suchstring{komplex} & 1.580 & 50 & 0 & 0 & 23.06.2016 \\
    \hline
    \suchstring{deployment} & 224.00 & 50 & 1 & 1 & 23.06.2016 \\
    \hline
    \suchstring{pipeline} & 80.800 & 50 & 2 & 1 & 23.06.2016 \\
    \suchstring{devops} & 1.230 & 50 & 5 & 1 & 23.06.2016 \\
    \hline
    \suchstring{einfach} & 3.180.000 & 50 & 0 & 0 & 23.06.2016 \\
    \hline
\end{tabular}\noindent
\\\\
Der Begriff \textit{Continuous Delivery} ist stark mit dem Bereich \underline{Biologie} verbunden. Gibt man diesen Begriff in Google Scholar ein, so sind die meisten Einträge aus der Biologie.
