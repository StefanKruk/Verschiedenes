% siehe: https://stackoverflow.com/questions/28608015/continuous-integration-vs-continuous-delivery-vs-continuous-deployment
\chapter{Zusammenfassung und Ausblick}
\label{chap:zusammenfassungUndAusblick}

\section{Zusammenfassung}
\label{sec:zusammenfassung}
Continuous Delivery ist ein großes Thema und durchaus eine Methode um die Produktivität zu verbessern. Produkte oder Funktionen nach einem bestimmten Zeitplan zu releasen ist nicht mehr möglich. Durch den sehr schnell wandelnden Markt gewinnt bei Unternehmen der Begriff Time-to-Market eine immer größere Bedeutung. Vor allem wenn es sich um Software- bzw. Internet-Unternehmen handelt. Durch stark fortschreitende Technik, können neue Produkte oder Funktionen deutlich schneller released werden, als je zu vor, wodurch bei den Unternehmen der Wettbewerbsdruck immer größer wird. Nur mit einer geringen Time-to-Market Zeitspanne ist es möglich, als Unternehmen dabei zu bleiben, denn auch Nutzer können sich meistens durch die große Menge an Angeboten deutlich schneller für ein anderes Produkt entscheiden, als je zu vor. Ein fester Release-Zeitpunkt für alle neuen Produkte bzw. Funktionen kann nicht nur dafür sorgen, dass Konkurrenzunternehmen ein Produkt schneller auf den Markt bringen als man selbst, es kann auch dazu führen, dass Funktionen im schlimmsten Fall, bei verpassen eines Release-Termins erst mit dem nächsten Release ausgerollt werden. Zudem ist es schwierig Fehler zu beheben, welche im Produktivsystem aufgetaucht sind.
\\\\
Continuous Delivery kann die Time-to-Market Zeitspanne deutlich verkürzen, wenn das Verfahren richtig angewendet wird, sowie eine frühere Erkennung der Fehler gewährleisten. Dazu ist es jedoch erforderlich einige Vorkehrungen zu treffen. Zum Beispiel müssen Tests möglichst weitgehend automatisiert werden und wiederholbar gemacht werden. Dadurch kann auch bei Änderungen am Code überprüft werden, ob diese Änderungen das Produkt beschädigen oder alle Funktionalitäten wie gewünscht weiterhin funktionieren. Zudem bekommt der Entwickler dadurch ein schnelles Feedback über seine Änderungen.
\\\\
Ziel von Continuous Delivery ist es, alle Schritte während der Entwicklung in eine sogenannte \textit{Continuous Delivery Pipeline} zu bringen, damit sie gemeinsam durchlaufen werden können. Dadurch ist es schließlich möglich von einem Festen Release-Termin zu \textit{"'Wann immer man möchte"'} Release zu wechseln, da immer ein Releasefähiges Artefakt entsteht; sofern die Pipeline erfolgreich durchlaufen wurde. Wird die Pipeline nicht erfolgreich durchlaufen, enthält der Entwickler ein direktes Feedback über den Fehler, da die Pipeline beim ersten Fehler unterbrochen wird. Dadurch können Fehler schon frühzeitig erkannt und behoben werden.

\section{Kritische Reflektion}
\label{sec:kritischeRelfektion}
Zu Beginn dieser Arbeit, war zwar klar in welche Richtung sich die Arbeit entwickeln sollte, jedoch wurden die Leitfragen noch nicht gründlich genug herausgearbeitet. Die Systematische Literaturrecherche hingegen bereitete weniger Probleme. Durch die in \cite{Kitchenham2007} angegebenen Quellen (siehe \ref{subsec:quellen} \nameref{subsec:quellen}) war es schnell möglich die einzelnen Quellen zu bewerten und einige relevante für die Recherche zu nehmen. Auch ergaben die Begriffe nach denen gesucht wurden und den Suchstrings keinerlei Probleme. Es stellte sich jedoch nach erster suche in den ausgewählten Quellen heraus, das nicht alle dieser Quellen geeignet waren, um dieses Thema zu bearbeiten. Der Begriff Continuous Delivery wird auch oft mit Themen aus der Biologie verbunden und sowohl der Suchzusatz \textit{"`IT"'} oder \textit{"`Software"'} ergaben keinerlei Treffer für das Gebiet der Informatik, sondern für spezielle Software, die das Biologische Continuous Delivery betrachtet. Daher mussten die Suchmaschinen auf \textit{IEEEXplore} und \textit{Google Scholar} beschränkt werden. Jedoch verwies auch \textit{Google Scholar} zum Teil auf Dokumente, die der Biologie angehören.
\\\\
Wie bereits erwähnt waren die Leitfrage nicht von Anfang an klar definiert, sodass der Hauptteil (Kapitel \ref{chap:continuousDelivery}) zunächst einmal nur eine grobe Fassung über Continuous Delivery war. Erst mit der Zeit entwickelte sich ein klares Bild, in welche Richtung sich die Arbeit wirklich entwickeln sollte. Ein Aspekt der hier mit spielt, ist die Tatsache das die Begriffe Continuous Integration/Delivery/Deployment nicht eindeutig definiert sind und sich je nach Definition nicht von einander unterscheiden. Lediglich der Begriff Continuous Integration unterscheidet sich im geringen von Continuous Delivery/Deployment und hat dadurch die Möglichkeit geboten, die zugrunde legende Strategie genauer zu erläutern und abzugrenzen. Aus diesen Gründen wurde der Begriff explizit genannt und erläutert. Der Begriff bzw. der Vergleich mit Continuous Deployment hingegen wurde weggelassen, da, wie bereits geschrieben, die Begriffe sich nicht eindeutig von einander unterscheiden.
\\\\
Da das Thema Continuous Delivery noch nicht so verbreitet ist und auch wenig Material, im Vergleich zu anderen Themen, bietet, wurden leider im Rahmen der durchgeführten Systematischen Literaturrecherche keine nennenswerte Studien über dieses Thema gefunden. Eine normale Google Suche ergab leider auch keine nennenswerten Studien zu diesem Thema.

\section{Ausblick}
\label{sec:Ausblick}
Diese Arbeit bietet einen kleinen Einblick in das Thema Continuous Delivery. Darauf aufbauend können zum Beispiel die Kosten für die Einführung von Continuous Delivery, sowie die Kostenunterschiede zwischen Entwicklungsprozesse mit und ohne Continuous Delivery ermittelt werden. Zudem kann explizit auf die Umstellung eines bestehenden Entwicklungsprozesses zu Continuous Delivery eingegangen werden. Dazu könnte zum Beispiel ein Prozess analysiert und bewertet werden und Schritt für Schritt in detaillierter Form festgehalten werden. Eine weitere Möglichkeit wäre es die Werkzeuge für Continuous Delivery genauer zu erläutern. Insbesondere das \textit{Continuous Integration Werkzeug Jenkins} könnte weiter erläutert werden. Es könnte jedoch auch ein Plugin für dieses Werkzeug geschrieben werden, wodurch die Abbildung einer Continuous Delivery Pipeline vereinfacht wird.