% siehe: https://stackoverflow.com/questions/28608015/continuous-integration-vs-continuous-delivery-vs-continuous-deployment
\chapter{Zusammenfassung und Ausblick}
\label{chap:zusammenfassungUndAusblick}

\section{Zusammenfassung}
\label{sec:zusammenfassung}
Continuous Delivery ist ein großes Thema. Dies sieht man an der zuvor genannten \nameref{sec:problemstellung}. Es wurde ein konkretes Beispiel genannt und dessen Probleme herausgearbeitet. Danach wurden die Leitfragen dieser Ausarbeitung geklärt. Als erstes sollte geklärt werden, was "'Continuous Delivery"' ist. Danach wurde der Begriff "'Continuous Delivery Pipeline"' eingeführt und darauf aufbauend der Entwicklungsprozess, des in der Problemstellung genannten Beispiels, in Continuous Delivery überführt.
\\\\
Zunächst wurde jedoch eine Systematische Literaturrecherche durchgeführt. Dafür wurden Inhaltliche Auswahl- und Ausschlusskriterien aufgestellt. Danach wurde mit Hilfe des PICOC-Ansatzes zunächst die Begriffe und der Personenkreis für die eigentliche Suche ermittelt. Außerdem wurden zusätzliche Anmerkungen zur Recherche, bezogen auf diese Ausarbeitung,  getroffen. Es wurden die eigentlichen Suchstrings gebaut und die zu durchsuchenden Quellen eingeführt. Darauf aufbauend wurden die Ergebnisse der Recherche wiedergegeben (Das Rechercheprotokoll ist im Anhang zu finden). Es stellte sich heraus, dass "'Google Scholar"' zwar viele Ergebnisse liefert, jedoch nicht viele relevante. Außerdem ist anzumerken, dass die wenigen Relevanten Ergebnisse der "'Google Scholar"'-Suche auf IEEExplore Dokumente verweisen, welche bereits bei der suche in dem genannten Archiv gefunden worden sind und daher nicht weiter betrachtet wurden. Die Ergebnisse der Systematischen Literaturrecherche wurden zwar Berücksichtigt und durchgelesen, jedoch umfassten sie im nach hinein betrachtet zu weitgehende Informationen, sodass sie nicht zum erläutern der Grundlagen dienen konnten und nur die im Literaturverzeichnis angegebenen Quellen verwendet wurden.
\\\\
Aufbauend auf die Systematische Literaturrecherche wurde das Thema "'Continuous Delivery"' behandelt. Dafür wurde zunächst der Begriff "'Continuous Delivery"' erläutert und es kristallisierte sich heraus, dass Continuous Integration ein großer Bestandteil von "'Continuous Delivery"' ist. Daher wurde zunächst Continuous Integration erläutert und die Teilprozesse kurz erwähnt. Darauf aufbauend wurde die "'Continuous Delivery Pipeline"' eingeführt und mit Hilfe von Continuous Integration und Delivery erläutert. Mit Hilfe der Pipeline wurde anschließend das Beispiel aus der Problemstellung um Continuous Delivery erweitert.
\\\\
Zuletzt erfolgt nun eine kritische Reflektion des Themas und ein Ausblick.

\section{Kritische Reflektion}
\label{sec:kritischeRelfektion}
Continuous Delivery ist ein Großes und für Unternehmen interessantes Thema. Es kann dafür sorgen, dass die Zeitspanne von der Idee bis zur Produktion (die Time-to-Market (TTM) Zeitspanne) verkürzt wird. Dadurch können Unternehmen sehr viel Geld sparen und Umsetzungen viel schneller in die Produktion bringen. Zusätzlich kann man den Build- und die Test-Prozesse durch Continuous Delivery standardisieren und automatisieren. Durch standardisierte Tests kann sichergestellt werden, dass auch nach Änderung einer Software, die Funktionalitäten weiterhin funktionieren. Außerdem bekommen Entwickler dadurch eine schnelle Rückmeldung über den Status der Software. Sie können direkt sehen, ob die Software ohne Fehler gebaut werden konnte und alle Tests erfolgreich durchgelaufen sind oder ob Fehler aufgetreten sind. Durch Automatisierte Tests kann außerdem sichergestellt werden, dass die Fehler in Produktion deutlich verringert werden, da die gleichen Tests ausgeführt werden können, welche vor einer Änderung erfolgreich ausgeführt wurden. Zudem kann der Prozess beliebig oft mit dem gleichen Einstellungen durchlaufen werden und das gleiche Ergebnis erwarten. Der Prozess ist also Idempotent.
\\\\
Jedoch muss \textit{Continuous Delivery} auch kritisch betrachtet werden. Zum einen sind nicht alle Tests automatisierbar. Zum Beispiel ist ein Oberflächentest sehr schwer automatisierbar. Zum anderen kann es schwer sein ein bestehenden Entwicklungsprozess um Continuous zu erweitern. Zusätzlich werden dafür ggf. neue Werkzeuge wie ein "'Build-management-Tool"' oder einen sogenannten Continuous Integration Server wie Jenkins benötigt. Entwickler müssen sich zunächst einmal mit diesen Werkzeugen auseinander setzten und erlernen. Manch ein Werkzeug ist recht teuer und durch das fehlende Wissen, kann es schnell passieren, dass falsche oder zu teure Software gekauft wird. Möchte ein Unternehmen also Continuous Delivery einführen, hat jedoch selber keine Ahnung des Themas, ist es ratsam einen Experten dazu zu holen, der einen beraten kann und ggf. Schulungen durchführen kann. Weiter sollten möglichst alle, aber auf jeden Fall ein Großteil der Entwickler davon überzeugt sein oder überzeugt werden können eine Veränderung des Entwicklungsprozesse zuzulassen bzw. durchzuführen.
\\\\
Es sei hier noch erwähnt, dass der Vergleich zwischen Continuous Integration vs Delivery vs Deployment ausgelassen wurde, da diese Begriffe nicht eindeutig definiert sind und zum Teil synonym zueinander verwendet werden. Es gibt keine direkten Vergleiche außerhalb von Blogs, jedoch wird für jeden klar, der sich mehr mit diesem Thema beschäftigt, das zum Teil die Übergänge fließend sind. Hier sollte sich jeder ein eigens Bild machen und sich selber mit dem Thema beschäftigen.

\section{Ausblick}
\label{sec:ausblick}
Continuous Delivery ist, wie bereits erwähnt, ein Großes Thema, daher konnte in dieser Ausarbeitung nur eine kleine Einführung des Themas stattfinden. Es kann jedoch noch zusätzlich Continuous Integration, Delivery und Deployment verglichen werden und die Werkzeuge erläutert werden, die zum Umsetzten des jeweiligen Prozesses benötigt werden. Man kann auf verschiedene Teilgebiete von Continuous Delivery, wie zum Beispiel dem "'Build-management-Tool"' eingegangen werden. Genauso kann man auf das "'Build und Management"' Werkzeug Jenkins eingegangen werden.
\\\\
Außerdem ist es möglich auf die Unterschiede in der Time-to-Market Zeitspanne zwischen einem Entwicklungsprozess ohne Continous Delivery und einen mit Continous Delivery einzugehen und darauf aufbauend die Kosten der jeweiligen Prozesse zu vergleichen.